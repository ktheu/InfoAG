%\documentclass[landscape,twocolumn,a4paper]{article}
\documentclass[a4paper,11pt,landscape,twocolumn]{book}
%\documentclass[a4paper]{article}

\usepackage[ngerman]{babel}
\usepackage[utf8]{inputenc}
\usepackage{amsmath}
\usepackage{amssymb}
\usepackage{listings} 
\usepackage{mathtools}
\usepackage{ulem}
\usepackage{eurosym}
\usepackage{textcomp}
\usepackage{soul}
\usepackage[top=20mm,left=10mm,right=10mm,bottom=10mm]{geometry}
\usepackage[german=quotes]{csquotes} 

\usepackage{fancyhdr}
\pagestyle{fancy}
\fancyhead[L]{Merkhilfe P5JS }
\fancyhead[R]{\thepage}
\fancyfoot{}
\setlength{\parindent}{0em} 

\fancypagestyle{ErsteSeite}{
   \fancyhf{}
   \fancyhead[L]{Merkhilfe - einfach}
   \fancyhead[R]{v04.02.2019}
} 
\begin{document}
\parskip 4pt
\thispagestyle{ErsteSeite}
\footnotesize
\lstset{tabsize=4, basicstyle=\footnotesize, showstringspaces=false,mathescape=true}
\lstset{literate=%
  {Ö}{{\"O}}1
  {Ä}{{\"A}}1
  {Ü}{{\"U}}1
  {ß}{{\ss}}1
  {ü}{{\"u}}1
  {ä}{{\"a}}1
  {ö}{{\"o}}1
}

%--------------------
\begin{lstlisting} 
// einzeiliger Kommentar,  /* mehrzeiliger Kommentar */ 
let x;        #  Deklaration
let x = 2;    #  Deklaration und Zuweisung

# Allerlei
x = x + 1;   # Zuweisung
$===$,   $!==$         # Gleichheit, Ungleichheit
&&, ||, !,   # boolesche Operatoren
true, false

# setup und draw
function setup() {             
  createCanvas(400, 400);
}
function draw() {
  background(220);
} 

# Systemvariablen
width, height    # wie in createCanvas gesetzt
mouseX, mouseY,  # mousePosition
frameCount       # Anzahl draw-Durchgänge

# Bedingte Anweisungen
if (x > width) {
   x = 0;
}
else {
   x = x + 1;
}
  
# for-Schleife
for (let i = 0; i < 20; i++) {   # 20 zufällige Punkte
   point(random(0, 300), random(0, 300));
}

# Zeichnen
point(x,y);
line(x1,y1,x2,y2);
rect(x,y,breite,hoehe);
ellipse(x,y,durchmesser,durchmesser);

# Farben
background(100);         # Grauwert
background(255,100,30);  # RGB-Wert
background(farbe), stroke(farbe), fill(farbe);  
strokeWeight(4); noStroke(); noFill();


# Zufall
randomSeed(42);
x = random(4,9) # x in [4,9)
x = int(random(4,9))  # ganze Zahl in [4,9)

# Text 
textSize(16);
text('Bitte eine Zahl eingeben', 20, 40);

# Events
function mousePressed() {
  if (mouseButton $===$ LEFT) { ... }
}

function keyPressed() {
  if (keyCode $===$ RIGHT_ARROW) { ... }
  if (key.toLowerCase() $===$ 'd') {...}
}
function keyReleased() {.... }  

# Keys
keyCode  // Variable mit Code des letzten gedrückten keys
UP_ARROW, DOWN_ARROW, LEFT_ARROW, RIGHT_ARROW.

# Functions
function lala(x, y) {
   let z = x + y * 2
   return z;
}

# Arrays
let a = [];      # leeres Array
a=[-1,2,4];      # mit Werten initialisiertes Array
a.push(17);      # Array erweitern
a[4] = 3;        # Array an einer Stelle updaten
for (let i = 0; i < a.length; i++) {   # Array durchlaufen
	a[i] = a[i] + 4;
}

\end{lstlisting} 

\end{document}