%\documentclass[landscape,twocolumn,a4paper]{article}
\documentclass[a4paper,11pt,landscape,twocolumn]{book}
%\documentclass[a4paper]{article}

\usepackage[ngerman]{babel}
\usepackage[utf8]{inputenc}
\usepackage{amsmath}
\usepackage{amssymb}
\usepackage{listings} 
\usepackage{mathtools}
\usepackage{ulem}
\usepackage{eurosym}
\usepackage{textcomp}
\usepackage{soul}
\usepackage[top=20mm,left=10mm,right=10mm,bottom=10mm]{geometry}
\usepackage[german=quotes]{csquotes} 

\usepackage{fancyhdr}
\pagestyle{fancy}
\fancyhead[L]{Merkhilfe P5JS }
\fancyhead[R]{\thepage}
\fancyfoot{}
\setlength{\parindent}{0em} 

\fancypagestyle{ErsteSeite}{
   \fancyhf{}
   \fancyhead[L]{Merkhilfe}
   \fancyhead[R]{v16.11.2018}
} 
\begin{document}
\parskip 4pt
\thispagestyle{ErsteSeite}
\footnotesize
\lstset{tabsize=4, basicstyle=\footnotesize, showstringspaces=false,mathescape=true}
\lstset{literate=%
  {Ö}{{\"O}}1
  {Ä}{{\"A}}1
  {Ü}{{\"U}}1
  {ß}{{\ss}}1
  {ü}{{\"u}}1
  {ä}{{\"a}}1
  {ö}{{\"o}}1
}

%--------------------
\begin{lstlisting} 
// einzeiliger Kommentar,  /* mehrzeiliger Kommentar */ 
let x;  const x = 10;  // Deklaration, Zuweisung 

# Allerlei
** Exponentation, i++;  i+=2;
$===$,   $!==$         // Gleichheit, Ungleichheit
&&, ||, !,   // boolesche Operatoren
true, false

# setup und draw
function setup() {             
  createCanvas(400, 400);
  // createCanvas(windowWidth,windowHeight);   
}
function draw() {
  background(220);
} 

# Systemvariablen
width, height    # wie in createCanvas gesetzt
mouseX, mouseY, pmouseX, pmouseY
frameCount      # Anzahl draw-Durchgänge
mouseIsPressed, keyIsPressed (irgendeiner)
keyIsDown(UP_ARROW)

# Sonstige Funktionen
dist(x1,y1,x2,y2)   # Abstand  
# Zeichnen
point(x,y);
line(x1,y1,x2,y2);
rect(x,y,breite,hoehe);
rectMode(CORNER/CENTER/CORNERS);
ellipse(x,y,xdurch,ydurch);
ellipseMode(CENTER/CORNER/CORNERS);
triangle(x1,y1,x2,y2,x3,y3);
quad(x1,y1,....);

# Farben
background(0); 
stroke(100); strokeWeight(4);
noStroke();
fill(grauwert); fill(r,g,b); fill(r,g,b,alpha);
noFill();
colorMode(HSB,360,100,100);
colorMode(RGB,255,255,255);
var farbe = '#D7F052';  
background(farbe), stroke(farbe), fill(farbe);

# Zufall
randomSeed(99);
x = random()      # Dezimalzahl x in [0,1)  
x = random(4)    # x in [0,4)
x = random(4,9) # x in [4,9)
x = random(a)    # a array zur Auswahl

# Text 
textAlign(LEFT); textSize(16);
text('Bitte eine Zahl eingeben', 20, 40);

# Events
function mousePressed() { ... }
function mouseReleased() {...}
function mouseClicked(){...}  
function keyPressed() {... }    # manche Browser bei key 
function keyReleased() {.... }  # nur Grossbuchstaben
function keyTyped() {... # auch Kleinbuchstaben}

# Keys
keyCode  // Variable mit Code des letzten gedrückten keys
BACKSPACE, DELETE, ENTER, RETURN, TAB, ESCAPE, SHIFT,
CONTROL, OPTION, ALT, UP_ARROW, DOWN_ARROW, 
LEFT_ARROW, RIGHT_ARROW.

# Functions


# Arrays
let a = [];
let a = new Array(20).fill(0);
a.push("Hi");  a=[-1,2,4];
a[4] = 3; 
a.toString(),  a.join(" * ");
x = a.pop();  # letztes Element
k = a.push(y); # y dranhängen, k neue Länge
x = a.shift();   # 1.Element wird gelöscht und zurückgegeben
k = a.unshift(x);  # fügt x vorne ein, k neue Länge
a[a.length] = 7;  # etwas dranhängen
a[10] = 7;          # ggf. mit Leerstellen was dranhängen
delete a[0];        # lässt vorne undefined Leerstelle  
a.splice(i,n,x1,x2,..);  # i Einfügeindex, n weg, x1,x2 ... rein
a.splice(0, 1);     # löscht erstes Element ohne undefinded Loch
c = a.concat(a1,a2,...)   # konkatenieren
b = a.slice(2)     # Teilarray ab Index 2
b = a.slice(2,5)  # Teilarray mit Index [2,5)
matrix=[ [0,1,2], [10,11,12] ];

# Sortieren von Arrays
a.sort()   #  die Elemente werden als Strings sortiert
a.reverse()
a.sort(function(x,y) {return abs(x) - abs(y)})  # für zahlen

# P5 Array-Funktionen
max(a), min(a)

# Ein Array durchlaufen
for (let i = 0; i < a.length; i++) {print(a[i]);}
a.forEach(m); function m(x) {print(x);}

# ein zweidimensionales Array 20x20:
let tmp = []; let size = 20;
while(size--) {tmp.push([]);}
      
# Dictionaries
spielkarte={farbe:"Pik", wert:8};

# Klassen
class Auto {
	constructor(farbe, kW) {
		this.farbe=farbe;
		this.kW=kW;
	}
	zeigeFarbe() {
		print("Die Farbe ist " + this.farbe);
	}
}
class BestandsAuto extends Auto {
	constructor(farbe, kW, preis)	{
		super(farbe, kW);
		this.preis=preis;
	}
	rabattiere(prozent)	{
		this.preis=this.preis*(100-prozent)/100;
	}
}
\end{lstlisting} 

\end{document}